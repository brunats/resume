% !TEX program = xelatex

\documentclass{resume}
%\usepackage{zh_CN-Adobefonts_external} % Simplified Chinese Support using external fonts (./fonts/zh_CN-Adobe/)
%\usepackage{zh_CN-Adobefonts_internal} % Simplified Chinese Support using system fonts

\usepackage{hyperref}

\begin{document}
\pagenumbering{gobble} % suppress displaying page number

\name{Bruna Tavares Silva}

\basicInfo{
  \email{silvatavares.bruna@gmail.com} \textperiodcentered\
  \phone{(+55) 47 9 99418932} \textperiodcentered\
  \linkedin[brunats]{https://https://www.linkedin.com/in/brunats}}

\section{\faGraduationCap\ Educação}
\datedsubsection{\textbf{Centro de Ciências Tecnológicas da UDESC (CCT - UDESC)}, Joinville, SC}{2014 -- Present}
\textit{Estudante de Graduação} em Ciência da Computação (BCC), prev. Dezembro 2018
\datedsubsection{\textbf{IFC Campus Araquari}, Araquari, SC}{2010 -- 2013}
\textit{Técnico} em Informática Integrado ao Ensino Médio

\section{\faUsers\ Experience}

\datedsubsection{\textbf{GRAUTH Project}}{Set. 2015}
\role{Ruby, HTML, JavaScript, CSS, Linux, Heroku, Github}{collaborated with Marlon Henry Schweigert}
Brief introduction: Uma aplicação em Ruby on Rails a qual utiliza autenticação com o Google,
\begin{itemize}
  \item Repositório GitHub: https://github.com/schweigert/grauth
\end{itemize}

\datedsubsection{\textbf{AG Evolutivo}}{Mar. 2018 -- Ago. 2018}
\role{Algoritmo evolutivo desenvolvido na Graduação}{Individual Projects}
Um algoritmo evolutivo com rotinas diversas de crossover, mutação, seleção e geração.
\begin{itemize}
  \item Implementação de rotinas:\begin{itemize}
    \item Crossover: Um ponto, Uniforme, BLX, PMX, Average;
    \item Mutação: BitFlip, Rand Mutation, Delta Mutation, Gaussian Mutation, Swap Positions;
    \item Generation Gab and Escalonamento Linear;
    \item Seleção: Roleta ou Torneio;
    \item Codificação: Inteiro, Binário e Float.
  \end{itemize}
  \item Repositório GitHub: https://github.com/brunats/computacao\_evolucionaria
\end{itemize}

\section{\faCogs\ Skills}
\begin{itemize}[parsep=0.5ex]
  \item Linguagens de Programação: C == Python = HTML5 = PHP = CSS3 > C++ = Ruby
  \item Platform: Linux, Windows, GitHub
  \item Development: Web, CRUD,
  \item DB: MySQL
\end{itemize}

\section{\faInfo\ Miscellaneous}
\begin{itemize}[parsep=0.5ex]
  \item GitHub: https://github.com/brunats
  \item Linkedin: https://www.linkedin.com/in/brunats
  \item Languages: English - Intermediário, Espanhol - Básico
\end{itemize}

%% Reference
%\newpage
%\bibliographystyle{IEEETran}
%\bibliography{mycite}
\end{document}
